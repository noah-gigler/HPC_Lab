\documentclass[unicode,11pt,a4paper,oneside,numbers=endperiod,openany]{scrartcl}

\input{assignment.sty}

\begin{document}


\setassignment
\setduedate{11 March 2024, 23:59}

\serieheader{High-Performance Computing Lab for CSE}{2024}
            {Student: FULL NAME}
            {Discussed with: FULL NAME}{Solution for Project 1a}{}
\newline

\assignmentpolicy

\section{Euler warm-up [10 points]}

1. The module system is a tool used to manage software environments on a Euler. 
It allows users to configure their environment by dynamically loading or unloading software modules. 
These modules adjust system variables to ensure that the necessary binaries and libraries are accessible.
You use it by loading specific software versions with module load and unloading them with module unload when done.
\\
2. Slurm is a tool used in big computer clusters to help manage who gets to use the computers and when. 
It schedules tasks and makes sure everything runs smoothly by allocating resources like processors and memory.
It's like the traffic controller for a cluster of computers.
\\


\section{Performance characteristics [50 points]}

\subsection{Peak performance}

\subsection{Memory Hierarchies}

\subsubsection{Cache and main memory size}

\subsection{Bandwidth: STREAM benchmark}

\subsection{Performance model: A simple roofline model}


\end{document}
