\documentclass[unicode,11pt,a4paper,oneside,numbers=endperiod,openany]{scrartcl}

\input{assignment.sty}

\usepackage{amsmath}

\setlength{\parindent}{0pt}

\begin{document}


\setassignment
\setduedate{11 March 2024, 23:59}

\serieheader{High-Performance Computing Lab for CSE}{2024}
            {Student: Noah Gigler}
            {Discussed with: FULL NAME}{Solution for Project 1a}{}
\newline

\assignmentpolicy

\section{Euler warm-up [10 points]}

1. The module system is a tool used to manage software environments on a Euler. 
It allows us to configure their environment by dynamically loading or unloading software modules. 
These modules adjust system variables to ensure that the necessary binaries and libraries are accessible.
You use it by loading specific software versions with module load and unloading them with module unload when done.
\\
2. Slurm is a tool used in big computer clusters to help manage who gets to use the computers and when. 
It schedules tasks and makes sure everything runs smoothly by allocating resources like processors and memory.
It's like the traffic controller for a cluster of computers.
\\
3.
see hostname.cpp
\\
4.
see slurm\_job\_one.sh
\\
5.
see slurm\_job\_two.sh

\section{Performance characteristics [50 points]}

\subsection{Peak performance}

\begin{table}[ht]
\centering
\caption{Euler VII Phase 1 and Phase 2 Specifications}
\begin{tabular}{|l|c|c|c|c|}
\hline
\textbf{Phase} & \textbf{Compute Nodes} & \textbf{CPUs per Node} & \textbf{CPU} & \textbf{Clock Speed (GHz)} \\ \hline
Phase 1        & 292                    & 2                      & AMD EPYC 7H12 & 2.6                        \\ \hline
Phase 2        & 248                    & 2                      & AMD EPYC 7763 & 2.45                       \\ \hline
\end{tabular}
\end{table}

Source: \url{https://scicomp.ethz.ch/wiki/Euler#Euler_VII_.E2.80.94_phase_1}

\begin{align*}
n_{\mathrm{super}} &= \frac{1}{TP} = 2 \\
n_{\mathrm{FMA}} &= 2 \\
n_{\mathrm{SMID}} &= 4 \\
\end{align*}

Values are the same for both Euler VII Phase 1 and 2.

Source for FMA, TP and : \url{https://uops.info/table.html} \\
Source for the SIMD values: "Software Optimization Guide for AMD EPYC™ 7002 Processors" and "Software Optimization Guide for AMD EPYC™ 7003 Processors"\\

\begin{align*}
P_{\mathrm{core}} &= n_{\mathrm{super}} \cdot n_{\mathrm{FMA}} \cdot n_{\mathrm{SMID}} \cdot f \\
P_{\mathrm{CPU}} &= P_{\mathrm{core}} \cdot \text{{\#Cores}} \\
P_{\mathrm{node}} &= P_{\mathrm{core}} \cdot \text{{\#CPUs}} \\
P_{\mathrm{Euler VII}} &= P_{\mathrm{node}} \cdot \text{{\#Nodes}}
\end{align*}

\begin{table}[ht]
\centering
\caption{Peak Performance Comparison}
\begin{tabular}{|l|c|c|}
\hline
\textbf{Metric} & \textbf{Phase 1} & \textbf{Phase 2} \\ \hline
$P_{\mathrm{core}}$ & $41.6 \, \mathrm{GFLOP/s}$ & $39.2 \, \mathrm{GFLOP/s}$ \\ \hline
$P_{\mathrm{CPU}}$ & $2.66 \, \mathrm{TFLOP/s}$ & $2.51 \, \mathrm{TFLOP/s}$ \\ \hline
$P_{\mathrm{node}}$ & $5.32 \, \mathrm{TFLOP/s}$ & $5.02 \, \mathrm{TFLOP/s}$ \\ \hline
$P_{\mathrm{Euler VII}}$ & $1.55 \, \mathrm{PFLOP/s}$ & $1.24 \, \mathrm{PFLOP/s}$ \\ \hline

\end{tabular}
\end{table}

\subsection{Memory Hierarchies}

\subsubsection{Cache and main memory size}

\begin{table}[ht]
\centering
\caption{Cache and Main Memory Sizes}
\begin{tabular}{|l|c|c|c|c|}
\hline
\textbf{Phase} & \textbf{L1 (KB)} & \textbf{L2 (KB)} & \textbf{L3 (MB)} & \textbf{Main Memory (GB)} \\ \hline
Phase 1        & 32               & 512              & 16               & 256                      \\ \hline
Phase 2        & 32               & 512              & 32               & 256                      \\ \hline
\end{tabular}
\end{table}

The only difference between the cache sizes of Phase 1 and Phase 2 is the L3 cache size. 
Phase 2 has twice the L3 cache size of Phase 1.

\subsection{Bandwidth: STREAM benchmark}

\begin{table}[ht]
\centering
\caption{Memory Bandwidth Comparison}
\begin{tabular}{|l|c|c|c|c|}
\hline
\textbf{Function} & \textbf{AMD EPYC 7H12} & \textbf{AMD EPYC 7763} \\ \hline
Copy              & 33044.7 MB/s           & 34873.5 MB/s           \\ \hline
Scale             & 25137.2 MB/s           & 24681.9 MB/s           \\ \hline
Add               & 25676.8 MB/s           & 25473.4 MB/s           \\ \hline
Triad             & 26346.1 MB/s           & 25710.9 MB/s           \\ \hline
\end{tabular}
\end{table}

As we can see the bandwidths are roughly the same for both processors and most operations.
The Copy operation has the highest bandwidth just as in the example provided. 
Aside from that nothing else is particularly interesting about the results.
The maximum bandwidth is approximately 25GB/s for both processors.


\subsection{Performance model: A simple roofline model}

The formula for the naive roofline model is:

$$
P = \begin{cases}
    \pi & \pi < \beta \cdot I \\
    \beta \cdot I & \text{else}
\end{cases}
$$

Where $\pi$ is the peak performance and $I$ is the operational intensity and $\beta$ is the peak bandwidth.
Here we are looking at the performance of a single core so we can use the peak performance we calculated in task 2.1.
The peak bandwidth was calculated in task 2.3. 

\end{document}
